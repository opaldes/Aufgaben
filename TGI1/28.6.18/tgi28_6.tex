% !TEX TS-program = pdflatex
% !TEX encoding = UTF-8 Unicode

% This is a simple template for a LaTeX document using the "article" class.
% See "book", "report", "letter" for other types of document.

\documentclass[11pt]{article} % use larger type; default would be 10pt

\usepackage[utf8]{inputenc} % set input encoding (not needed with XeLaTeX)

%%% Examples of Article customizations
% These packages are optional, depending whether you want the features they provide.
% See the LaTeX Companion or other references for full information.

%%% PAGE DIMENSIONS
\usepackage{geometry} % to change the page dimensions
\geometry{a4paper} % or letterpaper (US) or a5paper or....
% \geometry{margin=2in} % for example, change the margins to 2 inches all round
% \geometry{landscape} % set up the page for landscape
%   read geometry.pdf for detailed page layout information

\usepackage{graphicx} % support the \includegraphics command and options

% \usepackage[parfill]{parskip} % Activate to begin paragraphs with an empty line rather than an indent

%%% PACKAGES
\usepackage{booktabs} % for much better looking tables
\usepackage{array} % for better arrays (eg matrices) in maths
\usepackage{paralist} % very flexible & customisable lists (eg. enumerate/itemize, etc.)
\usepackage{verbatim} % adds environment for commenting out blocks of text & for better verbatim
\usepackage{subfig} % make it possible to include more than one captioned figure/table in a single float
% These packages are all incorporated in the memoir class to one degree or another...

%%% HEADERS & FOOTERS
\usepackage{fancyhdr} % This should be set AFTER setting up the page geometry
\pagestyle{fancy} % options: empty , plain , fancy
\renewcommand{\headrulewidth}{0pt} % customise the layout...
\lhead{}\chead{}\rhead{}
\lfoot{}\cfoot{\thepage}\rfoot{}

%%% SECTION TITLE APPEARANCE
\usepackage{sectsty}
\allsectionsfont{\sffamily\mdseries\upshape} % (See the fntguide.pdf for font help)
% (This matches ConTeXt defaults)

%%% ToC (table of contents) APPEARANCE
\usepackage[nottoc,notlof,notlot]{tocbibind} % Put the bibliography in the ToC
\usepackage[titles,subfigure]{tocloft} % Alter the style of the Table of Contents
\renewcommand{\cftsecfont}{\rmfamily\mdseries\upshape}
\renewcommand{\cftsecpagefont}{\rmfamily\mdseries\upshape} % No bold!

%%% END Article customizations

%%% The "real" document content comes below...

\title{TGI }
\author{Jeremy Seipelt}
%\date{} % Activate to display a given date or no date (if empty),
         % otherwise the current date is printed 

\begin{document}
\maketitle

\section{Geben Sie 4 Operatoren an, unter denen die reguläre Sprachen abgeschlossen sind.}
1. Vereinigung $\cup$\\
2. Schnitt $\cap$\\
3. Konkatenation $\bullet$\\
4. Differenz $\backslash$
\section{Automat A}
\subsection{Definieren Sie den Automaten als ein Quintupel}
$A = (\{q_{0},q_{1},q_{2},q_{3}\},\{0,1\},\delta_{A},q_{0},\{q_{3}\})$\\
$\delta_{A}: Q \times \sum \to Q$ mit \\
$\delta_{A}(q_{0}, 1) = q_{0} $\\
$\delta_{A}(q_{0}, 0) = q_{0} $\\
$\delta_{A}(q_{0}, 0) = q_{1} $\\
$\delta_{A}(q_{1}, 1) = q_{2} $\\
$\delta_{A}(q_{2}, 1) = q_{3} $\\
$\delta_{A}(q_{3}, 0) = q_{3} $\\
$\delta_{A}(q_{3}, 1) = q_{3} $\\
\subsection{Definieren Sie den Automaten A als ein DEA D}
Übergangstabelle für $\delta_{D}$\\
 \begin{tabular}{c|c|c}
  Q & 0 & 1 \\
  $q_{0} $&$ \{q_{0},q_{1}\}$ &$ q_{0}$\\
  $\{q_{0}, q_{1}\}$&$ \{q_{0},q_{1}\}$ &$\{ q_{0}, q_{2}\}$\\
  $\{q_{0}, q_{2}\}$&$ \{q_{0},q_{1}\}$ &$\{ q_{0}, q_{3}\}$\\
  $\{q_{0}, q_{3}\}$&$ \{q_{0},q_{1},q_{3}\}$ &$\{ q_{0}, q_{3}\}$\\
  $\{q_{0}, q_{1}, q_{3}\}$&$ \{q_{0},q_{1},q_{3}\}$ &$\{ q_{0}, q_{2}, q_{3}\}$\\
  $\{q_{0}, q_{2}, q_{3}\}$&$ \{q_{0},q_{1},q_{3}\}$ &$\{ q_{0}, q_{3}\}$\\
  $q_{1} $&$ \emptyset$ &$ q_{2}$\\
  $q_{2} $&$ \emptyset$ &$ q_{3}$\\
  $q_{3} $&$q_{3}$ &$ q_{3}$\\
 \end{tabular}
$D = (\{q_{0},q_{1},q_{2},q_{3}\},\{0,1\},\delta_{D},q_{0},\{q_{3}\})$\\

\end{document}
