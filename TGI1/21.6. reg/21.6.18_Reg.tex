% !TEX TS-program = pdflatex
% !TEX encoding = UTF-8 Unicode

% This is a simple template for a LaTeX document using the "article" class.
% See "book", "report", "letter" for other types of document.

\documentclass[11pt]{article} % use larger type; default would be 10pt

\usepackage[utf8]{inputenc} % set input encoding (not needed with XeLaTeX)

%%% Examples of Article customizations
% These packages are optional, depending whether you want the features they provide.
% See the LaTeX Companion or other references for full information.

%%% PAGE DIMENSIONS
\usepackage{geometry} % to change the page dimensions
\geometry{a4paper} % or letterpaper (US) or a5paper or....
% \geometry{margin=2in} % for example, change the margins to 2 inches all round
% \geometry{landscape} % set up the page for landscape
%   read geometry.pdf for detailed page layout information

\usepackage{graphicx} % support the \includegraphics command and options

% \usepackage[parfill]{parskip} % Activate to begin paragraphs with an empty line rather than an indent

%%% PACKAGES
\usepackage{booktabs} % for much better looking tables
\usepackage{array} % for better arrays (eg matrices) in maths
\usepackage{paralist} % very flexible & customisable lists (eg. enumerate/itemize, etc.)
\usepackage{verbatim} % adds environment for commenting out blocks of text & for better verbatim
\usepackage{subfig} % make it possible to include more than one captioned figure/table in a single float
% These packages are all incorporated in the memoir class to one degree or another...

%%% HEADERS & FOOTERS
\usepackage{fancyhdr} % This should be set AFTER setting up the page geometry
\pagestyle{fancy} % options: empty , plain , fancy
\renewcommand{\headrulewidth}{0pt} % customise the layout...
\lhead{}\chead{}\rhead{}
\lfoot{}\cfoot{\thepage}\rfoot{}

%%% SECTION TITLE APPEARANCE
\usepackage{sectsty}
\allsectionsfont{\sffamily\mdseries\upshape} % (See the fntguide.pdf for font help)
% (This matches ConTeXt defaults)

%%% ToC (table of contents) APPEARANCE
\usepackage[nottoc,notlof,notlot]{tocbibind} % Put the bibliography in the ToC
\usepackage[titles,subfigure]{tocloft} % Alter the style of the Table of Contents
\renewcommand{\cftsecfont}{\rmfamily\mdseries\upshape}
\renewcommand{\cftsecpagefont}{\rmfamily\mdseries\upshape} % No bold!

%%% END Article customizations

%%% The "real" document content comes below...

\title{Aufgabe zu regulären Ausdrücken}
\author{Jeremy Seipelt}
%\date{} % Activate to display a given date or no date (if empty),
         % otherwise the current date is printed 

\begin{document}
\maketitle

\section{Welche von den folgenden Wörtern gehören zur Sprache L(A)?}
Überprüfung durch manuelle Eingabe, akzeptiert falls Endzustand erreicht wird:
\subsection{A: 10 }
$q_{0}\rightarrow1\rightarrow q_{1}\rightarrow 0 \rightarrow *q_{3}$\\ 10 ist ein akzeptiertes Wort
\subsection{B: 1110 }
$q_{0} \rightarrow 1 \rightarrow q_{1} \rightarrow 1\rightarrow q_{2} \rightarrow 1\rightarrow q_{0}\rightarrow 0\rightarrow q_{1}$\\ 1110 ist ein nicht akzeptiertes Wort
\subsection{C: 0101 } 
$q_{0} 0 q_{1} 1 q_{2} 0 q_{0} 1 q_{1}$\\ 0101 ist ein nicht akzeptiertes Wort
\subsection{D: 01011 }
$q_{0}\rightarrow 0\rightarrow q_{1} \rightarrow 1 \rightarrow q_{2} \rightarrow 0\rightarrow q_{0} \rightarrow 1 \rightarrow q_{1} \rightarrow 1 \rightarrow q_{2}$\\ 01011 ist ein nicht akzeptiertes Wort
\section{Geben Sie reguäre Ausdrücke an für:}
Benutzt wurden reguläre Ausdrücke für Javascript
\subsection{a) groß geschriebene Worte wie z.B. Welt, aber nicht WELT.}
([A-Z])\textbackslash w+
\subsection{b) groß geschriebene Worte mit mindestens 3 und höchstens 5 Buchstaben.}
(([A-Z])(\textbackslash w)\{0,4\})\textbackslash b
\section{Zeigen oder widerlegen Sie:}
\subsection{b Wenn L eine reguläre Sprache ist und L = L1 $\cup$ L2 gilt, dann sind auch L1 und L2 regulär.}
Wenn $L_{1}$ und $L_{2}$ regulär sind so gibt es reguläre Ausdrücke R und  S, sodass  $L_{1}$=L(R) und  $L_{2}=L(S)$\\
Die Sprache der Vereinigung von R und S ist somit  L(R+S) = L(R)$\cup$ L(S) welche als regulär gilt.\\
Dadurch können wir folgende Gleichung erstellen L(R+S) = L(R)$\cup$ L(S) =  L1 $\cup$ L2\\
Die Sprache L1 $\cup$L2 ist  also regulär, da sie einen regulären Ausdruck hat.
\subsection{b Wenn L1, L2 und L3 regulär sind, dann auch L1$\bullet$(L2 $\cup$L3).}
Aufgabe b zeigt das die Vereinigung zweier regulärer Sprachen als eine Reguläresprache benutzt werden kann. Deshalb gilt L4 = L2 $\cup$L3\\
Da L1 und L2 regulär sind gibt es reguläre Ausdrücke R und S, so dass L1 = L(R) und L4 = L(S).  R$\bullet$S ist die Konkatenation der Sprachen
von R und S, deshalb gilt L(RS) = L(R).L(S).\\
Dadurch können wir folgende Gleichung aufstellen L(RS) = L(R).L(S) = L1.L4 =  L1$\bullet$(L2 $\cup$L3).\\
Die Sprache L1$\bullet$(L2 $\cup$L3) ist  also regulär, da sie einen regulären Ausdruck hat.
\end{document}
