% !TEX TS-program = pdflatex
% !TEX encoding = UTF-8 Unicode

% This is a simple template for a LaTeX document using the "article" class.
% See "book", "report", "letter" for other types of document.

\documentclass[11pt]{article} % use larger type; default would be 10pt

\usepackage[utf8]{inputenc} % set input encoding (not needed with XeLaTeX)

%%% Examples of Article customizations
% These packages are optional, depending whether you want the features they provide.
% See the LaTeX Companion or other references for full information.

%%% PAGE DIMENSIONS
\usepackage{geometry} % to change the page dimensions
\geometry{a4paper} % or letterpaper (US) or a5paper or....
% \geometry{margin=2in} % for example, change the margins to 2 inches all round
% \geometry{landscape} % set up the page for landscape
%   read geometry.pdf for detailed page layout information

\usepackage{graphicx} % support the \includegraphics command and options

% \usepackage[parfill]{parskip} % Activate to begin paragraphs with an empty line rather than an indent

%%% PACKAGES
\usepackage{booktabs} % for much better looking tables
\usepackage{array} % for better arrays (eg matrices) in maths
\usepackage{paralist} % very flexible & customisable lists (eg. enumerate/itemize, etc.)
\usepackage{verbatim} % adds environment for commenting out blocks of text & for better verbatim
\usepackage{subfig} % make it possible to include more than one captioned figure/table in a single float
% These packages are all incorporated in the memoir class to one degree or another...

%%% HEADERS & FOOTERS
\usepackage{fancyhdr} % This should be set AFTER setting up the page geometry
\pagestyle{fancy} % options: empty , plain , fancy
\renewcommand{\headrulewidth}{0pt} % customise the layout...
\lhead{}\chead{}\rhead{}
\lfoot{}\cfoot{\thepage}\rfoot{}

%%% SECTION TITLE APPEARANCE
\usepackage{sectsty}
\allsectionsfont{\sffamily\mdseries\upshape} % (See the fntguide.pdf for font help)
% (This matches ConTeXt defaults)

%%% ToC (table of contents) APPEARANCE
\usepackage[nottoc,notlof,notlot]{tocbibind} % Put the bibliography in the ToC
\usepackage[titles,subfigure]{tocloft} % Alter the style of the Table of Contents
\renewcommand{\cftsecfont}{\rmfamily\mdseries\upshape}
\renewcommand{\cftsecpagefont}{\rmfamily\mdseries\upshape} % No bold!

%%% END Article customizations

%%% The "real" document content comes below...

\title{Brief Article}
\author{The Author}
%\date{} % Activate to display a given date or no date (if empty),
         % otherwise the current date is printed 

\begin{document}
\maketitle

\section{1 Zustandsautomaten aufstellen für 1 + (0 + 11)(1 + 01)*0}
Siehe im Archiv enthaltenes Graphik findia.svg
\section{2 Konkatenation von Sprachen}
\subsection{a  L1 = \{0, 1, 011\} L2 = \{$\epsilon$, 0, 1, 01\}}
L1$\bullet$L2 = \{0,1,011,00,10,0110,01,11,0111,001,101,01101\}\\
L2$\bullet$L1 = \{0,1,00,01,01011,10,11,0111,010,011,01011\}\\
\subsection{b L1 = \{$\epsilon$, a, ab, abb\} L2 = \{a, b, bb\}}
L1$\bullet$L2 = \{a,b,bb,aa,ab,abb,aba,abbb,abba,abbbb\}\\
L2$\bullet$L1 = \{a,b,bb,aa,aab,aabb,ba,bab,babb,bba,bbab,bbabb\}\\
\subsection{c L1 = \{$\epsilon$, -, +, -+\} L2 = \{$\epsilon$, -, +, +-\}}
L1$\bullet$L2 = \{-,+,+-,--,-+,++,++-,-+-,-++,-++-\}\\
L2$\bullet$L1 = \{-,+,-+,--,--+,+-.++.+-+,+--,+--+\}\\
\subsection{d L1= \{$\epsilon, a, ab,aba$\}  L2= \{$\epsilon,a,b,bab$\}}
Die eigentlichen Symbole der aufgabe wurden der einfachheit halber auf a und b geändert.
L1$\bullet$L2 = \{a,b,bab,aa,ab,abab,aba,abb,abbab,abaa,abab,ababab\}\\
L2$\bullet$L1 = \{a,ab,aba,aa,aab,aaba,ba,bab,baba,babab,bababa\}\\
 \section{Pumping Lemma}
 \subsection{L = \{$w = 0^{i}1^{j}| i \leq 2^{j}+1$\}} 
 Annahme L ist regulär.
 Dann kann ich mit Pumping Lemma ein Wort xyz bilden das ein Wort der Sprache L ist. 
 Pumpingzahl n = 3;
 Zeichenreihe x aus L bei der der $\vert x\vert$ $\geq$ n.
 Pumpen mit dem Wort 011 welches in der Sprache liegt.\\
 Zerlegeung in uvw:  u=0 v=1 w=1
Es gilt: \\ $\vert v\vert$ $\geq$ 1\\
Es gilt: \\  $\vert uv\vert$ $\leq$ n; n = 3\\
 1. Versuch abpumpen von v mit v$^{0}$:
 01 liegt nicht in der Sprache \\
 Da 1. Versuch fehlgeschlagen ist ist die Annahme die Sprache sei regulär falsch.
   
 \subsection{L = \{$w = 0^{i}1^{j}| i \neq j$\}}
 
 Annahme L ist regulär.
 Dann kann ich mit Pumping Lemma ein Wort xyz bilden das ein Wort der Sprache L ist. 
 Pumpingzahl n = 2;
 Damit Wort akzeptiert wird i = n und j= n+1, somit kann i niemals gleich j sein.
 Zeichenreihe x aus L bei der der  $\vert uv\vert$ $\geq$ n mit einsetzen von n.
 Pumpen mit dem Wort 00111 welches in der Sprache liegt.\\
 Zerlegeung in uvw:  u=0 v=0 w=111
Es gilt: \\  $\vert v\vert$ $\geq$ 1\\
Es gilt: \\  $\vert uv\vert$ $\leq$ n; n = 2\\
1. Versuch abpumpen von v mit v$^{0}$:
0111 liegt in der Sprache \\
2. Versuch pumpen von v mit v$^{2}$:
000111 liegt nicht in der Sprache, da  $\vert 0\vert$ =  $\vert 1\vert$ ist. Wiederspruch im Pumping Lemma , deswegen ist L nicht Regulär.\\

\end{document}
