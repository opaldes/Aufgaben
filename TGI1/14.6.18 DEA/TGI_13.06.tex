% !TEX TS-program = pdflatex
% !TEX encoding = UTF-8 Unicode

% This is a simple template for a LaTeX document using the "article" class.
% See "book", "report", "letter" for other types of document.

\documentclass[11pt]{article} % use larger type; default would be 10pt

\usepackage[utf8]{inputenc} % set input encoding (not needed with XeLaTeX)

%%% Examples of Article customizations
% These packages are optional, depending whether you want the features they provide.
% See the LaTeX Companion or other references for full information.

%%% PAGE DIMENSIONS
\usepackage{geometry} % to change the page dimensions
\geometry{a4paper} % or letterpaper (US) or a5paper or....
% \geometry{margin=2in} % for example, change the margins to 2 inches all round
% \geometry{landscape} % set up the page for landscape
%   read geometry.pdf for detailed page layout information

\usepackage{graphicx} % support the \includegraphics command and options

% \usepackage[parfill]{parskip} % Activate to begin paragraphs with an empty line rather than an indent

%%% PACKAGES
\usepackage{booktabs} % for much better looking tables
\usepackage{array} % for better arrays (eg matrices) in maths
\usepackage{paralist} % very flexible & customisable lists (eg. enumerate/itemize, etc.)
\usepackage{verbatim} % adds environment for commenting out blocks of text & for better verbatim
\usepackage{subfig} % make it possible to include more than one captioned figure/table in a single float
% These packages are all incorporated in the memoir class to one degree or another...

%%% HEADERS & FOOTERS
\usepackage{fancyhdr} % This should be set AFTER setting up the page geometry
\pagestyle{fancy} % options: empty , plain , fancy
\renewcommand{\headrulewidth}{0pt} % customise the layout...
\lhead{}\chead{}\rhead{}
\lfoot{}\cfoot{\thepage}\rfoot{}

%%% SECTION TITLE APPEARANCE
\usepackage{sectsty}
\allsectionsfont{\sffamily\mdseries\upshape} % (See the fntguide.pdf for font help)
% (This matches ConTeXt defaults)

%%% ToC (table of contents) APPEARANCE
\usepackage[nottoc,notlof,notlot]{tocbibind} % Put the bibliography in the ToC
\usepackage[titles,subfigure]{tocloft} % Alter the style of the Table of Contents
\renewcommand{\cftsecfont}{\rmfamily\mdseries\upshape}
\renewcommand{\cftsecpagefont}{\rmfamily\mdseries\upshape} % No bold!

%%% END Article customizations

%%% The "real" document content comes below...

\title{TGI HA}
\author{Jeremy Seipelt}
%\date{} % Activate to display a given date or no date (if empty),
         % otherwise the current date is printed 

\begin{document}
\maketitle

\section{Sind die gegebenen Automaten deterministisch oder nichtdeterministisch? Warum?}

Die beiden gegebenen Automaten A$_{1}$ und A$_{2}$ sind nach erstem betrachten deterministisch.
Es gibt keine sichtbare Eingabe die den Übergang in unterschiedliche Zustände verursacht.
Um meine Annahmen zu Beweisen werde ich eine Übergangstabelle für die Automaten anlegen.
Solllte sich in einer Zelle der Tabelle zwei Werte(Zustände) befinden ist der Automat nicht deterministisch und damit meine Annahme falsch.\\
Bei der Tabelle ist zu beachten dass es zu einigen Ungereimtheiten kommt, es folgt ein Beispiel anhand von A$_{1}$.
Die Menge der Eingaben $\alpha \neq \{h\}$ das Element $t$ enthält aber beim Zustand  $q_{1}$ unterschiedliche Übergänge. 
Bei solchen Widersprüchen gilt, für die folgende Aufgabe, dass das genannte Element($t$) nichtmehr zur Menge($\alpha \neq \{h\}$) gehört.
\section{Geben Sie die Menge der Zustände $Q_{1}$ und $Q_{2}$ für beide Automaten an?}
\subsection{$Q_{1}=\{q_{0},q_{1},q_{2},q_{3}\}$}
\subsection{$Q_{2}=\{q_{0},q_{1},q_{2},q_{3},q_{4},q_{5}\}$}
\section{Stellen Sie die Übergangstabellen für beide Automaten auf }
\subsection{Automat A$_{1}$}
\begin{tabular}{c|c|c|c|c|c|c}
 $\delta$ & $\alpha \neq $\{h\}&  $\alpha \neq $\{t\}  &  $\alpha \neq $\{w\} & $\alpha = $\{h\}&  $\alpha = $\{t\}  &  $\alpha = $\{w\}\\
 \hline
 $\rightarrow q_{0}$ & $q_{0}$  & $q_{0}$ & $q_{0}$ & $q_{1}$ & $q_{0}$ & $q_{0}$ \\
 \hline
 $q_{1}$ & $q_{0}$  &  $q_{0}$ &  $q_{0}$ &  $q_{1}$ &  $q_{2}$ &  $q_{0}$ \\
 \hline
 $q_{2}$ &  $q_{0}$  &  $q_{0}$ &  $q_{0}$ &  $q_{1}$ &  $q_{0}$ &  $q_{3}$ \\
\hline
 $\ast q_{3}$ &  $q_{3}$  &  $q_{3}$ &  $q_{3}$ &  $q_{3}$ &  $q_{3}$ &  $q_{3}$ \\
 \hline
 \end{tabular}\\
\subsection{Automat A$_{2}$}
\begin{tabular}{c|c|c|c|c|c|c}
 $\delta$ & $\alpha \neq $\{t\}&  $\alpha \neq $\{e\}  &  $\alpha \neq $\{h\} & $\alpha = $\{t\}&  $\alpha = $\{e\}  &  $\alpha = $\{h\}\\
\hline
$\rightarrow q_{0}$ & $q_{0}$  & $q_{0}$ & $q_{0}$ & $q_{1}$ & $q_{0}$ & $q_{0}$ \\
\hline
$q_{1}$ & $q_{0}$  &  $q_{0}$ &  $q_{0}$ &  $q_{1}$ &  $q_{4}$ &  $q_{2}$ \\
\hline
$q_{2}$ &  $q_{0}$  &  $q_{0}$ &  $q_{0}$ &  $q_{1}$ &  $q_{3}$ &  $q_{0}$ \\
\hline
$\ast q_{3}$ &  $q_{3}$  &  $q_{3}$ &  $q_{3}$ &  $q_{3}$ &  $q_{3}$ &  $q_{3}$ \\
\hline
$q_{4}$ &  $q_{0}$  &  $q_{0}$ &  $q_{0}$ &  $q_{1}$ &  $q_{5}$ &  $q_{0}$ \\
\hline
$\ast q_{5}$ &  $q_{5}$  &  $q_{5}$ &  $q_{5}$ &  $q_{5}$ &  $q_{5}$ &  $q_{5}$ \\
 \hline
 \end{tabular}\\
\section{Geben Sie die Mengen der finalen Zustände $F_{1}$ und $F_{2}$ an. Sind die Mengen der
finalen Zuständen $F2_{1}$ und $F2_{2}$ echte Teilmengen der Mengen $Q_{1}$ und $Q_{2}$?}
\subsection{$F_{1}=\{q_{3}\}$}
\begin{tabular}{ll}
$F_{1}\subseteq Q_{1}$ & $F_{1}$ ist eine Teilmenge von $Q_{1}$\\
$Q_{1}=\{q_{0},q_{1},q_{2},q_{3}\} \neq F_{1}$ & $F_{1}$ ist ungleich $Q_{1}$\\
\hline
$F_{1}\subset Q_{1}$ & $F_{1}$ ist eine echte Teilmenge von $Q_{1}$
\end{tabular}
\subsection{$F_{2}=\{q_{3},q_{5}\}$}
\begin{tabular}{ll}
$F_{2}\subseteq Q_{2}$ & $F_{2}$ ist eine Teilmenge von $Q_{2}$\\
$Q_{2}=\{q_{0},q_{1},q_{2},q_{3},q_{4},q_{5}\} \neq F_{1}$ & $F_{1}$ ist ungleich $Q_{1}$\\
\hline
$F_{2}\subset Q_{2}$ & $F_{2}$ ist eine echte Teilmenge von $Q_{2}$
\end{tabular}
\end{document}
